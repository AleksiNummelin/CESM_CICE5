%=======================================================================
% SVN: $Id: ice_what_is_cice4.tex 5 2005-12-12 17:41:05Z mvr $
%=======================================================================

This User's Guide accompanies the CCSM4 User's Guide, and is intended
for those who would like to run CICE coupled, on a supported
platform, and "out of the box".  Users running CICE fully coupled should first
look at the CCSM4 User's Guide.  It includes a quick start guide for downloading
the CCSM4 source code and input datasets, and information on how to configure,
build and run the model.  The supported configurations and scripts for building
the fully coupled model are also described in the CCSM4 User's Guide.  The CICE
User's Guide is intended for users interested in making modifications to the
ice model scripts or namelists or running the uncoupled ice model.  Users
interested in modifying the source code should see the CICE Code Reference/
Developer's Guide.

CICE4 is the latest version of the Los Alamos Sea Ice Model, sometimes
referred to as the Community Ice CodE.  It is the
result of a community effort to develop a portable, efficient sea ice model
that can be run coupled in a global climate model or uncoupled as a stand-alone
ice model. It has been released as the sea ice component of the Community
Climate System Model (CCSM), a fully-coupled global climate model that
provides simulations of the earths past, present and future climate states.
CICE4 is supported on high- and low-resolution Greenland Pole
and tripole grids, which are identical to those used by the Parallel Ocean 
Program (POP) ocean model.  The high resolution version is best suited for 
simulating present-day and future climate scenarios while the low resolution 
option is used for paleoclimate simulations and debugging.  An uncoupled 
version of CICE is available separately from Los Alamos National Laboratory:\\

\ifpdf
                    {\tt http://oceans11.lanl.gov/trac/CICE}.\\
\else
\begin{htmlonly}
  \htmladdnormallink{http://oceans11.lanl.gov/trac/CICE}
                    {http://oceans11.lanl.gov/trac/CICE}.
\end{htmlonly}
\begin{latexonly}
                    {\tt http://oceans11.lanl.gov/trac/CICE}.
\end{latexonly}
\fi

It provides a means of running the sea ice model independent of the other CCSM 
components.  It reads in atmospheric and ocean forcing, which eliminates the 
need for the flux coupler, and the atmosphere, land and ocean data models.  It 
can be run on a reduced number of processors, or without MPI (Message Passing 
Interface) for researchers without access to these computer resources.

The physics in the uncoupled ice model are identical to those in the
ice model used in the fully coupled system.  CICE is a dynamic-thermodynamic
model that includes a subgrid-scale ice thickness distribution (\cite{bitz01};
\cite{lips01}).  It uses the energy conserving thermodynamics 
of \cite{bitz99}, has multiple layers in each thickness category, and accounts
for the influences of brine pockets within the ice cover.  The ice dynamics
utilizes the elastic-viscous-plastic (EVP) rheology of \cite{hunk97}.  Sea ice
ridging follows \cite{roth75b} and \cite{thor75}.  A slab ocean mixed layer model
is included.  A Scientific Reference is available that contains more detailed
information on the model physics. 

An attempt has been made throughout this document to provide the following
text convention.  Variable names used in the code are {\tt typewritten}.
Subroutine names are given in {\it italic}, and file names are in {\bf boldface}.

\subsection{What's new in CICE4?}

CICE4 is an upgraded version of the Community Sea Ice Model, CSIM5, which was
 based on CICE3, and was released in June 2004.  The model physics are similar to that of CSIM5, but it was decided to move to CICE, the LANL sea ice model for practical reasons.  The major changes are:

\begin{itemize}

\item The incremental remapping transport scheme is now the default and is
      available in the modules called {\bf ice\_transport\_driver.F90} and 
      {\bf ice\_transport\_remap.F90}.  The MPDATA transport scheme,
      is no longer supported in CICE4. The upwind advection scheme is the
      only additional option and is contained in 
      {\bf ice\_transport\_driver.F90}.

\item The standalone ice model is now only available through Los Alamos National
      Laboratory.

\item Several physics options have been shifted around into other or new
      modules. For example, most of {\bf ice\_albedo.F90} is now in 
      {\bf ice\_shortwave.F90}. The new module contains all of the shortwave
      radiative transfer plus the basic albedo calculations.

\item The mechanical redistribution scheme has been changed significantly and
      is available in {\bf ice\_mechred.F90}.

\item A new drivers area has been created for modules that are specific
      to the CCSM as opposed to the standalone CICE model. The new CCSM
      drivers are contained in the cpl\_mct and cpl\_share subdirectories.
      The ESMF driver (cpl\_esmf) is still under development. The source
      subdirectory now contains driver independent source code for the
      most part.

\item A new bld subdirectory has been introduced which contains CCSM
      specific build and configure scripts. These scripts handle the
      namelist generation, defaults, and configuration details.

\end{itemize}

The CICE source code is based on the Los Alamos sea ice model CICE model version 4.  The main source code is very similar in both versions, but the drivers are significantly different.  If there are some topics that are not covered in the CICE documentation, users are encouraged to look at the CICE documentation \cite{cice08}.  It is available at Los Alamos National Laboratory at: \\
\\

\ifpdf
                    {\tt http://oceans11.lanl.gov/trac/CICE}.
\else
\begin{htmlonly}
  \htmladdnormallink{http://oceans11.lanl.gov/trac/CICE}
                    {http://oceans11.lanl.gov/trac/CICE}.
\end{htmlonly}
\begin{latexonly}
                    {\tt http://oceans11.lanl.gov/trac/CICE}.
\end{latexonly}
\fi
