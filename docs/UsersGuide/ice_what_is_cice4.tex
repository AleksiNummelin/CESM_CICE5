%=======================================================================
% CVS: $Id: ice_what_is_cice4.tex 5 2005-12-12 17:41:05Z mvr $
% CVS: $Source$
% CVS: $Name$
%=======================================================================

This User's Guide accompanies the CCSM4 User's Guide, and is intended
for those who would like to run CICE coupled or uncoupled, on a supported
platform, and "out of the box".  Users running CICE fully coupled should first
look at the CCSM4 User's Guide.  It includes a quick start guide for downloading
the CCSM4 source code and input datasets, and information on how to configure,
build and run the model.  The supported configurations and scripts for building
the fully coupled model are also described in the CCSM4 User's Guide.  The CICE
User's Guide is intended for users interested in making modifications to the
ice model scripts or namelists or running the uncoupled ice model.  Users
interested in modifying the source code should see the CICE Code Reference/
Developer's Guide.

CICE4 is the latest version of the Los Alamos Sea Ice Model, sometimes
referred to as the Community Ice CodE.  It is the
result of a community effort to develop a portable, efficient sea ice model
that can be run coupled in a global climate model or uncoupled as a stand-alone
ice model. It has been released as the sea ice component of the Community
Climate System Model (CCSM), a fully-coupled global climate model that
provides simulations of the earths past, present and future climate states.
CICE4 is supported on high- and low-resolution Greenland Pole
and tripole grids, which are identical to those used by the Parallel Ocean 
Program (POP)
ocean model.  The high resolution version is best suited for simulating 
present-day and future climate scenarios while the low resolution option is
used for paleoclimate simulations and debugging.

An uncoupled version of CICE is available separately from Los Alamos National
Laboratory (http://oceans11.lanl.gov/trac/CICE).  It provides a means of running
the sea ice model independent of the other CCSM components.  It reads in
atmospheric and ocean forcing, which eliminates the need for the flux coupler,
and the atmosphere, land and ocean data models.  It can be run on a reduced
number of processors, or without MPI (Message Passing Interface) for researchers
without access to these computer resources.

The physics in the uncoupled ice model are identical to those in the
ice model used in the fully coupled system.  CICE is a dynamic-thermodynamic
model that includes a subgrid-scale ice thickness distribution
(\cite{bitz01};\cite{lips01}).  It uses the energy conserving thermodynamics
of \cite{bitz99}, has multiple layers in each thickness category, and accounts
for the influences of brine pockets within the ice cover.  The ice dynamics
utilizes the elastic-viscous-plastic (EVP) rheology of \cite{hunk97}.  Sea ice
ridging follows \cite{roth75b} and \cite{thor75}.  A slab ocean mixed layer model
is included.  A Scientific Reference is available that contains more detailed
information on the model physics. 

An attempt has been made throughout this document to provide the following
text convention.  Variable names used in the code are {\tt typewritten}.
Subroutine names are given in {\it italic}, and file names are in {\bf boldface}.

\subsection{What's new in CICE4?}

CICE4 is an upgraded version of the Community Sea Ice Model, CSIM5, which was
 based on CICE3, and was released in October 2002.  
The model physics are similar to that of CSIM5, but it was decided to move to
CICE, the LANL sea ice model for practical reasons.  The major changes are:

\begin{itemize}

\item A module for a new incremental remapping transport scheme was added
      called {\bf ice\_transport\_remap.F}.  The MPDATA transport scheme,
      formerly in {\bf ice\_transport.F},  was moved to {\bf ice\_transport\_mpdata.F}.
      Open water advection was added to the incremental remapping.

\item A bug in {\bf ice\_albedo.F} was fixed to avoid negative albedos for thin,
      bare, melting ice.

\item A bug in {\bf ice\_ocean.F} was fixed to include {\tt fswthru} in the calculation
      of sea surface temperature.

\item A salt flux calculation was added so the ice reference salinity could
      be changed to a non-zero value.

\item The sea ice momentum equation modified for the free drift regime.  The
      dynamics scheme treats areas with lower ice concentrations more accurately.
      See \cite{hunk03}.

\item {\bf ice\_coupling.F} has been rewritten to be compatible with the latest
      version of the CCSM coupler.

\item An additional field {\tt Qref} is calculated in {\it atmo\_boundary\_layer}
      and passed to the coupler.

\item Each ice thickness category has 4 thickness layers. Previously, the two thinnest
      categories had two layers.

\item A sub-cycling timestep {\tt ndyn\_dt} was added to the dynamics to get around
      a model instability that would manifest itself in MPDATA.

\item The snow and ice albedos, used for coupled model tuning, were moved to the
      namelist to make modification easier.

\item It is now possible to run CICE as an uncoupled model.  The module
      {\bf ice\_flux\_in.F} has been added to read in forcing data.

\item Most modules have been modified to run efficiently on vector platforms.
      Grid indices {\tt i}, {\tt j} are no longer passed into subroutines.  Shorter
      loops over ice categories and vertical layers have been moved outside
      the longer loops over {\tt i} and {\tt j}.  Directives have been placed
      before certain loops to enforce vectorization.

\item The thermodynamics modules from CICE4, {\bf ice\_tstm.F}, {\bf ice\_vthermo.F},
      {\bf ice\_therm\_driver.F}, and {\bf ice\_dh.F}, have been replaced by two new
      modules {\bf ice\_therm\_vertical.F} and {\bf ice\_therm\_itd.F}.  The caclulations
      in {\bf ice\_therm\_vertical.F} are done before the {\it to\_coupler} call,
      and those in {\bf ice\_therm\_itd.F} are done after this call.

\item New modules {\bf ice\_exit.F} and {\bf ice\_work.F} have been added that
      contain code for aborting the model and globally accessible work arrays.

\item The gx3v4 grid had been replaced by a new gx3v5 grid. The coupled model
      produced a poor meridional overturning circulation (MOC) with the gx3v4 grid.
      The new grid has points in different locations and has higher resolution
      in the North Atlantic than gx3v4.  The simulation with gx3v5 gives a better ice
      thickness distribution and produces a better MOC.

\item The prescribed ice model is not supported in this release.

\item Most CICE and CICE modules are very similar, except for the mechanical
      redistribution modules.

\end{itemize}

The CICE source code is based on the LOS Alamos sea ice model CICE model.
After a code merger with CICE was carried out to take advantage of the vector-
friendly code, the models are very similar.  If there are some topics that
are not covered in the CICE documentation, users are encouraged to look
at the CICE documentation \cite{cice04}.  It is available at \\
\begin{htmlonly}
  \htmladdnormallink{http://climate.lanl.gov/Models/CICE/index.htm}
                    {http://climate.lanl.gov/Models/CICE/index.htm}.
\end{htmlonly}
\begin{latexonly}
                    {\tt http://climate.lanl.gov/Models/CICE/index.htm}.
\end{latexonly}

