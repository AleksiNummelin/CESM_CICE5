%=======================================================================
% CVS: $Id: ice_build.tex 5 2005-12-12 17:41:05Z mvr $
% CVS: $Source$
% CVS: $Name$
%=======================================================================

\subsection{The Build Environment}

The coupled and uncoupled ice models use the Makefile and make environment.
These files are located in the {\bf models/ice/cice/src/drivers/ccsm_sequential}
directory for the coupled model and in {\bf models/ice/cice/src/drivers/cice4}
for the uncoupled model. These directories contain the following files:

\begin{itemize}
\item  {\bf Macros.AIX} contains build settings specific to AIX (IBM SP3).
\item  {\bf Macros.IRIX64} contains build settings specific to IRIX64 (SGI Origin).
\item  {\bf makdep.c} evaluates the code dependencies for the source code
\item  {\bf Makefile} is a generic gnumakefile
\end{itemize}

There is a {\bf Macros.$<$OS$>$} file for each supported platform. These files contain
machine dependent preprocessor, compiler and library information for building the model.
The {\bf Macros.$<$OS$>$} files for the uncoupled model have been simplified, since most of
the libraries used by the coupled model are not used by the uncoupled ice model. 
If you are running the model on a platform other than those tested, you
will need to create a new {\bf Macros.$<$OS$>$} file and modify the paths and settings
for your system.  In some cases, CICE has a set of options that are different from
the default values at the top of the file.  These are after the line
{\tt ifeq (\$(MODEL),cice)} in the {\bf Macros.$<$OS$>$} files and are described below.

\subsubsection{CICE Preprocessor Flags}

Preprocessor flags are activated in the form {\tt -Doption} in the {\bf Macros.$<$OS$>$}
files.  The flags specific to the ice model are

\begin{verbatim}
CPPDEFS :=  $(CPPDEFS) -Dcoupled -DNPROC_X=$(NX) -DNPROC_Y=$(NY) -D_MPI
\end{verbatim}

The option {\tt -Dcoupled} is set to activate the coupling interface.  This 
will include the source code in {\bf ice\_coupling.F}, for example.  For uncoupled
runs, it has been removed.
If a coupler other than the CCSM coupler is used, there is a flag called
{\tt -Dfcd\_coupled} that will keep the fluxes from being divided by the ice area.
In coupled runs, the CCSM coupler multiplies the fluxes by the ice area, so
they are divided by the ice area in CICE to get the correct fluxes.

The options {\tt -DNPROC\_X=\$(NX)} and {\tt -DNPROC\_Y=\$(NY)} set the number of
processors used in each grid direction.  These values are set automatically
in the scripts for the coupled model, and in {\bf cice\_run} by the user for
uncoupled runs.  {\tt NX} and {\tt NY} must divide evenly into the grid, and
are used only for MPI grid decomposition.  If {\tt NX} or {\tt NY} do not divide evenly
into the grid, the model setup will exit from the setup script
and print an error message to the {\bf ice.log*} (standard out) file.

The flag {\tt -D\_MPI} sets up the message passing interface.  This must be set
for runs using a parallel environment.  To get a better idea of what code
is included or excluded at compile time, grep for {\tt ifdef} and {\tt ifndef}
in the source code or look at the {\bf *.f} files in the /{\bf obj} directory.

\subsubsection{CICE Compiler Options}

The name of the Fortran compiler is set by the variable {\tt FC} in the
{\bf Macros.$<$OS$>$} files.  The default name of the compiler is {\tt f90}
on the SGI and {\tt mpxlf90\_r} on the IBM SP.  CICE uses the following compiler
options on the SGI platform:

\begin{verbatim}
   FFLAGS     := -c -64 -mips4 -O2 -r8 -i4 -show -extend_source
\end{verbatim}

On the IBM, the following compiler options are set in {\bf Macros.AIX}:

\begin{verbatim}
   FFLAGS  := -c -O2 -qstrict -Q -qmaxmem=-1 -qrealsize=8  \
              -qarch=auto -qtune=auto
\end{verbatim}

