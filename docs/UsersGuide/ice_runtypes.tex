%=======================================================================
% CVS: $Id: ice_runtypes.tex 5 2005-12-12 17:41:05Z mvr $
% CVS: $Source$
% CVS: $Name$
%=======================================================================
% This contains the description of initial, branch and continue runs

The run types available for the coupled model are described in the CCSM
User's Guide. There are two run types available for the uncoupled ice
model and are described in this section.

\subsection{Startup Runs}

If {\tt \$RUNTYPE} is set to {\tt startup}, the model will read in the restart file
that resides in {\tt \$CSIMDATA} called {\bf iced.0001-01-01.\$ICE\_GRID.20lay}.
The conditions in this file are for the gx1v3 grid and are from an equilibrium run
using modified NCEP forcing.  The setup script will create a pointer file named 
{\bf rpointer.ice} with the name of the initial restart file in it Startup
runs can also be initialized using data created within the ice model, as
described in the next section.

\subsubsection*{Using Initial Conditions from within CSIM}
\label{model_generated}

Initial conditions can be calculated within the ice model in a subroutine
called {\it init\_state} in {\bf ice\_init.F}.  Here, the ocean surface is initialized
at the freezing point everywhere north of 40 degrees and south of -40 degrees
latitude,  allowing ice to form everywhere in these regions.
While running the ice model with a data ocean will melt any extra
ice during the first year of integration, is not recommended that these
initial conditions be used when running the ice model coupled to an
active ocean model.  The advantage of using this input is that it is not
grid or land mask dependent.  To use these initial conditions, set {\tt \$RESTART}
in {\bf csim.setup.csh}:

\begin{verbatim}
set RESTART = .false.
\end{verbatim}

Initializing the model using a restart file from an equilibrium run
will result in a more physically reasonable scenario than the initial
conditions set within CSIM. The drawbacks are that the data is
binary, difficult to edit, and is date and grid dependent.
A restart file will be used as initial conditions if

\begin{verbatim} set RESTART = .true.  \end{verbatim}

{\noindent in {\bf csim.setup.csh}.}

\subsection{Continue Runs}
\label{continue_runs}

A continue run is an exact continuation of a previous run. This means
that the run will produce a bit-for-bit answer as if the existing run
had not stopped.  The input data file is determined by the filename set
in the restart pointer file (see section \ref{pointer_files}).
It is assumed that {\tt \$CASE} has not changed. For a continue run,
the only change required in the run script is to set:

\begin{verbatim} RUNTYPE    continue \end{verbatim}

The date will continue from the previous run, since it is read in from the
restart file.


