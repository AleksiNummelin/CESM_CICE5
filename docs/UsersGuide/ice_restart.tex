%=======================================================================
% CVS: $Id: ice_restart.tex 5 2005-12-12 17:41:05Z mvr $
% CVS: $Source$
% CVS: $Name$
%=======================================================================

Restart files contain all of the initial condition information
necessary to restart from a previous simulation.  These files are in a
standard IEEE 64 bit binary format. A restart file is not necessary for
an initial run, but is highly recommended.  The initial conditions
that are internal to the ice model produce an unrealistic ice cover
that an uncoupled ice model will correct in several years.  The initial
conditions from a restart file are created from an equilibrium solution,
and provide more realistic information that is necessary if coupling
to an active ocean model.  The frequency at which restart files are created
is controlled by the namelist parameter {\tt dumpfreq}. 
The names of these files are proceeded by the namelist parameter 
{\tt dump\_file} and, by default are written out yearly to the {\bf /rest}
directory under the executable directory.  To change the directory where these files are
located, modify the variable {\tt \$RSTDIR} at the top of the setup script.
The names of the restart files follow the CCSM Output Filename Requirements.
The form of the restart file names are as follows: \\

{\bf \$CASE.csim.r.yyyy-mm-dd-sssss} \\

For example, the file {\bf \$CASE.csim.r.0002-01-01-00000} would be written
out at the end of year 1, month 12.  A file containing the name of a restart
file is called a restart pointer file. This filename information allows the
model simulation to continue from the correct point in time, and hence the
correct restart file.

\subsubsection*{Restart Pointer Files}
\label{pointer_files}

A pointer file is an ascii file named {\bf rpointer.ice} that contains the
path and filename of the latest restart file. The model uses this information to find
a restart file from which initialization data is read.  The pointer files are
written to and then read from the executable directory. For 
{\tt startup} runs, a pointer is created by the ice setup script 
Whenever a restart file is written, the existing restart pointer file 
is overwritten.  The namelist variable {\tt pointer\_file} contains the
name of the pointer file. Pointer files seldom need editing.  The contents
are usually maintained by the setup script and the component model. 

