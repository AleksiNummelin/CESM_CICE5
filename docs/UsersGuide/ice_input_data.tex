%=======================================================================
% CVS: $Id: ice_input_data.tex 5 2005-12-12 17:41:05Z mvr $
% CVS: $Source$
% CVS: $Name$
%=======================================================================

Both the coupled and uncoupled CSIM require a minimum of three files to run:

\begin{itemize}
  \item {\bf global\_\$\{ICE\_GRID\}.grid} is a binary file containing
        grid information and is renamed  \\
        {\bf data.domain.grid}  \\
        when it is copied to the executable directory.
  \item {\bf global\_\$\{ICE\_GRID\}.kmt} is a binary file containing
        land mask information and is renamed \\
        {\bf data.domain.kmt}.
  \item {\bf iced.0001-01-01.\$\{ICE\_GRID\}.20lay} or 
        {\bf iced.0001-01-01.gx3v5.040213} are binary files containing
         initial condition information for the gx1v3 and gx3v5 grids,
         respectively. The thickness distribution in this
         restart file contains 5 categories, each with 4 layers.
\end{itemize}

Depending on the grid selected in the scripts, the appropriate {\bf global*}
and {\bf iced*} files will be copied and renamed in the executable directory.
Currently, only gx3v5 and gx1v3 grids are supported for the ice and ocean
models.\\

An additional data file is necessary to use the ocean mixed layer within the
ice model, depending on the specified grid:
\begin{itemize}
  \item {\bf pop\_frc\_gx1v3\_010815.nc} contains monthly averaged ocean forcing
        from POP output.
  \item {\bf pop\_frc\_gx3v5\_040212.nc} same as above, but for the gx3v5 grid.
\end{itemize}

This file is renamed {\bf oceanmixed\_ice.nc} when it is copied into the executable
directory.

\subsection{Atmospheric Forcing}
\label{atm_forcing}

The uncoupled ice model will run without atmospheric forcing.  It will use
the fluxes set in subroutine {\it init\_flux}.  For atmospheric forcing, the following
datasets are available on the gx3v5 grid, and can be read in using the module
{\bf ice\_flux\_in.F}.  The directory where the data is located will have to be set
in this module, and not in the scripts. The files are as follows: \\

\noindent{\bf /wherever/you/put/it/atm/gx3v5/ISCCPM/MONTHLY/RADFLX/cldf.1997.dat} \\
{\bf /wherever/you/put/it/atm/gx3v5/ISCCPM/MONTHLY/RADFLX/swdn.1997.dat} \\
{\bf /wherever/you/put/it/atm/gx3v5/MXA/MONTHLY/PRECIP/prec.1997.dat}    \\
{\bf /wherever/you/put/it/atm/gx3v5/NCEP/4XDAILY/STATES/dn10.1997.dat}   \\
{\bf /wherever/you/put/it/atm/gx3v5/NCEP/4XDAILY/STATES/q\_10.1997.dat}  \\
{\bf /wherever/you/put/it/atm/gx3v5/NCEP/4XDAILY/STATES/t\_10.1997.dat}  \\
{\bf /wherever/you/put/it/atm/gx3v5/NCEP/4XDAILY/STATES/u\_10.1997.dat} \\
{\bf /wherever/you/put/it/atm/gx3v5/NCEP/4XDAILY/STATES/v\_10.1997.dat} \\

These files are in direct access binary files, and the source is evident
from the path names.  {\bf cldf.1997.dat} and {\bf swdn.1997.dat} contain
the monthly averaged cloud fraction and downwelling shortwave.
{\bf prec.1997.dat} is the monthly averaged precipitation in mm/month.
The remaining files are the atmospheric density, specific humidity,
air temperature, and wind fields.
Note that these datasets are meant for testing the model and are
not the best observational data for research.  Users are advised not to
publish results based on these datasets.  To use this forcing, the
following lines in {\bf ice.F} need to be uncommented:

\begin{verbatim}
call init_getflux
call getflux
\end{verbatim}

